%%%%%%%%%%%%%%%%%%%%%%%%%%%%%%%%%%%%%%%%%%%%%%%%%%%%%%%%%%
%  Copyright (C) 2005 WanT group                         %
%  This file is distributed under the terms of the       %
%  GNU General Public License.                           %
%  See the file `License'  in the root directory of      %
%  the present distribution,                             %
%  or http://www.gnu.org/copyleft/gpl.txt                %
%%%%%%%%%%%%%%%%%%%%%%%%%%%%%%%%%%%%%%%%%%%%%%%%%%%%%%%%%%

\thispagestyle{empty}
\section{Installation procedure}\label{section:install}

\noindent \underline {NOTES}: (i) The present version of the code
adopts the installation procedure of PWSFC (for more details see
also www.pwscf.org). (ii) This installation procedure is still
experimental, and only a limited number of architectures are
currently supported.
Installation procedure is also reported in the {\bf docs/README.install} file.\\

\noindent Installation is a two-step procedure:
\begin{enumerate}
\item ``cd'' to the top directory of the \WANT\ tree (that should be
the one where this file is), and issue this command at the shell
prompt:\\
$./$configure
\item Now run:\\
     make target\\
\end{enumerate}
\noindent where ``target'' is one (or more) of the following:
disentangle, wannier, conductor, bands, blc2wan, plot. Running
"make" without arguments prints a list of available targets.\\

\noindent Cross-compilation is not currently supported.

%----------------------------------------------------------------------
%First step : configuring
%----------------------------------------------------------------------

\subsection{Step one: configuring} ``configure'' is a GNU-style configuration script,
automatically generated by GNU Autoconf.  (If you want to play
with it, its source file is ``conf/configure.ac''; you may also
want to edit ``conf/make.sys.in'')  It generates the following
files:
\begin{displaymath}
\begin{array}{ll}
  \textrm{\$TOPDIR/make.sys}        &      \textrm{compilation settings and flags}\\
  \textrm{\$TOPDIR/*/makedeps.sh}   &      \textrm{dependencies, per
  source directory}
\end{array}
\end{displaymath}
\noindent where \$TOPDIR is the top directory of the \WANT\ source
tree.\\

\noindent ``.dependencies'' files are actually generated by the
``makedeps.sh'' shell script.  If you modify the program sources,
you might have to rerun it.  Note that you must run it from the
directory it is in.\\

\noindent To force using a particular compiler, or compilation
flags, or libraries, you may set the appropriate environment
variables when running the configuration script.  For example:

\begin{description}
  \item ./configure CC=gcc CFLAGS=-O3 LIBS=``-llapack -lblas
  -lfftw''
\end{description}

\noindent Some of those environment variables are:
\begin{displaymath}
\begin{array}{ll}
  \textrm{TOPDIR}       &\textrm{: top directory of the \WANT\ tree (defaults to `pwd`)}\\
  \textrm{F90, F77, CC} &\textrm{: Fortran 90, Fortran 77, and C compilers}\\
  \textrm{CPP}          &\textrm{: source file preprocessor (defaults to "\$CC -E")}\\
  \textrm{LD}           &\textrm{: linker (defaults to \$F90)}\\
  \textrm{CFLAGS, FFLAGS,}  &\textrm{ }\\
  \textrm{F90FLAGS, CPPFLAGS, LDFLAGS} &\textrm{: compilation flags}\\
  \textrm{LIBDIRS}      &\textrm{: extra directories to search for libraries (see below)}\\
\end{array}
\end{displaymath}

\noindent You should always be able to compile the \WANT\ suite of
programs without having to edit any of the generated files.  If
you ever have to, that should be considered a bug in the
configuration script and you are encouraged to submit a bug
report.\\

\noindent \underline {IMPORTANT}: \WANT\ can take advantage of
several
optimized numerical libraries:\\
\noindent - ESSL on AIX systems (shipped by IBM)\\
\noindent - MKL together with Intel compilers (shipped by Intel,
free for non-commercial use)\\
\noindent  - ATLAS (freely downloadable from
http://math-atlas.sourceforge.net/)\\
\noindent  - FFTW (freely downloadable from
http://www.fftw.org/)\\

\noindent The configuration script attempts to find those
libraries, but may fail if they have been installed in
non-standard locations. You should look at the LIBS environment
variable (either in the output of the configuration script, or in
the generated ``make.sys'') to check whether all available
libraries were found.\\

\noindent If any libraries weren't found, you can rerun the
configuration script and pass it a list of directories to search,
by setting the environment variable LIBDIRS; directories in the
list must be
separated by spaces.  For example:\\

\begin{description}
  \item ./configure LIBDIRS=``/opt/intel/mkl/mkl61/lib/32
  /usr/local/lib/fftw-2.1.5''
\end{description}

\noindent If this still fails, you may set the environment
variable LIBS manually and retry.  For example:\\

\begin{description}
  \item ./configure LIBS=``-L/cineca/prod/intel/lib -lfftw -llapack
  -lblas''
\end{description}

\noindent Beware that in this case, you must specify *all* the
libraries that you want to link to.  The configuration script will
blindly accept the specified value, and will *not* search for any
extra libraries.\\

\noindent If you want to use the FFTW library, the ``fftw.h"
include file is also required.  If the configuration script wasn't
able to find it, you can specify the correct directory in the
INCLUDEFFTW environment variable. For example:\

\begin{description}
  \item ./configure INCLUDEFFTW=``/cineca/lib/fftw-2.1.3/fftw''
\end{description}

%----------------------------------------------------------------------
%Second step : compiling
%----------------------------------------------------------------------
\subsection{Step two: compiling}
\noindent Here is a list of available compilation targets:
\begin{displaymath}
\begin{array}{llll}
  \textrm{make all}        &  \textrm{compile} & \textrm{wannier/disentangle.x} &\textrm{(step 1)}\\
  \textrm{}                &  \textrm{}       & \textrm{wannier/wannier.x}     &\textrm{(step 2)}\\
  \textrm{}                &  \textrm{}       & \textrm{wannier/bands.x}       &\textrm{(post proc)}\\
  \textrm{}                &  \textrm{}       & \textrm{wannier/plot.x}        &\textrm{(post proc)}\\
  \textrm{}                &  \textrm{}       & \textrm{wannier/blc2wan.x}     &\textrm{(post proc)}\\
  \textrm{}                &  \textrm{}       & \textrm{transport/conductor.x} &\textrm{(step 3)}\\
  \textrm{make clean}      &  \textrm{remove}  & \textrm{Object files and
  executables}&  \textrm{}\\
    \textrm{make veryclean}      &  \textrm{remove}  & \textrm{Configuration files too}&  \textrm{}
\end{array}
\end{displaymath}

\noindent \underline {IMPORTANT}: If you change any compilation or
precompilation options after a previous (successful or failed)
compilation, you must run ``make clean'' before recompiling,
unless you know exactly which routines are affected by the changed
options and how to force their recompilation.

\subsection{List of directories}
Within the top directory of the \WANT\ tree there are the
following directories:

\newdimen\descindent \descindent = 8pc
{\noindent \leftskip = \descindent \parskip = .5\baselineskip
\llap{\hbox to \descindent{\bf conf\hfil}}%
includes the configuration and compilation files\par

\noindent\llap{\hbox to \descindent{\bf docs\hfil}}%
includes the documentation files and manuals \par

\noindent\llap{\hbox to \descindent{\bf iotk\hfil}}%
includes the files for the iotk input/output interface \par

\noindent\llap{\hbox to \descindent{\bf transport\hfil}}%
includes the source files and modules of for program conductor.x
\par

\noindent\llap{\hbox to \descindent{\bf wannier\hfil}}%
includes the source files and modules of for programs
disentangle.x, wannier.x, bands.x, plot.x, blc2wan.x \par

\noindent\llap{\hbox to \descindent{\bf bin\hfil}}%
includes links of all the executable file *.x \par

\noindent\llap{\hbox to \descindent{\bf include\hfil}}%
includes the environmental file *.h \par

\noindent\llap{\hbox to \descindent{\bf libs\hfil}}%
includes source file for internal libraries \par

\noindent\llap{\hbox to \descindent{\bf tests\hfil}}%
includes tutorial examples for the use of \WANT\ suite of
codes\par

\noindent\llap{\hbox to \descindent{\bf utility\hfil}}%
includes some useful tools for \WANT\ use. \par}
